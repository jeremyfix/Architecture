\bibliographystyle{plainnat}
\bibliography{biblio}
\nocite{*}

%% \begin{itemize}
%% \item \url{http://www.cs.umd.edu/class/sum2003/cmsc311/Notes/}
%% \item tableaux de Karnaugh ?
%% \item \url{http://www.ecs.umass.edu/ece/koren/#courses} : pas mal de choses sur l'architecture, des simulateurs java de cache , mémoire, etc.. et des choses sur l'arithmétique binaire ! et des simulateurs sur ces opérations \url{http://www.ecs.umass.edu/ece/koren/arith/simulator/}
%% \item supports du cours MIT 6.004 Chris Terman
%% \item simulateur javascript de circuits électroniques avec des transistors FET \url{https://github.com/6004x/jade}
%% \end{itemize}


%% Liste de quelques composants et leurs datasheets
%% \begin{itemize}
%% \item 8-input multiplexer : 74HC/HCT151
%% \item 3-to-8 decoder/demultiplexer :  74HC238
%% \end{itemize}

%% \begin{itemize}
%% \item \url{http://www.interfacebus.com/voltage_threshold.html} : graphe des Vih, vol, voh,.. et aussi une description des acronymes des composants pour savoir si ce sont des CMOS, TTL, ...
%% \item \url{https://learn.sparkfun.com/tutorials/logic-levels} : une présentation des niveaux logiques
%% \item toto
%% \end{itemize}

%Utilisation de circuitikz pour réaliser de jolis circuits en LaTex \url{http://www.latex-tutorial.com/tutorials/advanced/}, Lesson 12/14;


