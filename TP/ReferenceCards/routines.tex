\documentclass{standalone}
\standaloneconfig{border=10pt}
\usepackage[utf8]{inputenc}
\usepackage[french]{babel}
\usepackage{amsmath}
\usepackage{cases}
\usepackage{booktabs}

\begin{document}

\begin{minipage}{5in}
\setlength{\parindent}{10pt}
\setlength{\parskip}{3ex plus 0.5ex minus 0.2ex}

\textbf{\underline{Sous-programmes}}\\

L'appel et le retour de routines (sous-programmes) s'effectuent par les instructions CALL et RET.\\

L'instruction CALL doit sauvegarder le compteur de programme (PC) sur la pile avant de brancher à l'adresse fournie par l'opérande. L'instruction RET dépile le compteur de programme (PC).\\

\renewcommand{\arraystretch}{1.2}
\begin{tabular}{@{}lllll@{}}
\toprule
Nom & Mnémonique & Nb. d'arguments & Opcode \\
\toprule
Appel de routine & CALL & 1 & 0xa0 \\
Retour de routine & RET & 0 & 0xa8
\end{tabular}

\end{minipage}

\end{document}
