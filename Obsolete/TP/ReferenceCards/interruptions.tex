\documentclass{standalone}
\standaloneconfig{border=10pt}
\usepackage[utf8]{inputenc}
\usepackage[french]{babel}
\usepackage{amsmath}
\usepackage{cases}
\usepackage{booktabs}

\begin{document}

\begin{minipage}{5in}
\setlength{\parindent}{10pt}
\setlength{\parskip}{3ex plus 0.5ex minus 0.2ex}

\textbf{\underline{Interruptions}}\\

L'activation ou inactivation des interruptions dépendents du register \emph{Interrupt Flag} : elles sont activiées si $IF = 1$ et désactivées sinon. L'activation des interruptions s'effectue par l'instruction STI (\emph{Set Interrupt}) et l'inactivation par l'instruction CLI (\emph{Clear Interrupt}).\\

L'architecture proposée ne supporte qu'une interruption. L'appel de la routine d'interruption s'effectue par l'instruction INT et le retour de la routine d'interruption par l'instruction RTI. 

\renewcommand{\arraystretch}{1.2}
\begin{tabular}{@{}lllll@{}}
\toprule
Nom & Mnémonique & Nb. d'arguments & Opcode \\
\toprule
Inactivation des interruptions & CLI & 0 & 0xd0 \\
Activation des interruptions & STI & 0 & 0xd4\\
Appel de l'interruption & INT & 0 & 0xe0 \\
Retour de l'interruption & RTI & 0 & 0xe8
\end{tabular}

\end{minipage}

\end{document}
